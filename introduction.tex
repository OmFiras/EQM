\chapter{Introduction}
\label{chp:introduction}
Last years, regard to the great advances made in technology and their
influence on society, have been generated new trends and innovations in
surveillance systems. Such is the case of video security cameras on public
(parks, underground stations, airports), private (gambling houses, hotels,
banks), and restricted places (high reactive chemical labs, electrical
installations, radiation areas); that, with the target of maintaining a record
of events they were changing and improving over time. Currently, one can
have full access to high definition, infrared vision, and even biometric
identification systems. All aiming to address automation under government
regulations to protect and guarantee people's security and privacy.

In January 2012, a CCTV (Closed Circuit Television) captured the moment when a
woman is rescued after falling on to the tracks in an underground station in
Madrid. September 8th, also in 2012, a drunk man falls of an underground
platform, then gets robbed and hit by the train.
(Figure~\ref{fig:chp-intr:underground-sweden}) Hopefully, there were no further
consequences. Should be said that these kind of events does not always occur on
stations. For instance, accidents in motorways and main roads can be found
around the world. (see Figure~\ref{fig:chp-intr:car-accidents}) 

Note that, in last paragraph, the word \textit{event} was used to reference a
given situation described under a period of time on a geographical place.
Furthermore, is feasible the fact of classification events into
\textit{normal} and \textit{abnormal}, remaining to abnormal events be those
which have a very low likelihood of occurrence (a more detailed definition
of normal and abnormal events will be discussed in
Section~\ref{chp:related-work,sec:terms}). In this respect, people falling and
car crashing represent abnormal events.

These exposed cases, underground stations and car accidents in road ways, were
events were recorded by a video security camera. Although no information about
the presence of an automatic event detection system was found, those videos are
likely to be used only forensically. The addition of analytics can be able
to alert security personnel of possible or potential situations where people's
safety is on risk.

\begin{figure}
 \centering
 \includegraphics[width=5cm,height=3cm]
 {/Introduction/underground-station-a-1-s.png}
 \includegraphics[width=5cm,height=3cm]
 {/Introduction/underground-station-a-2-s.png}
 \includegraphics[width=5cm,height=3cm]
 {/Introduction/underground-station-a-3-s.png}
 \includegraphics[width=5cm,height=3cm]
 {/Introduction/underground-station-a-5-s.png}
 \includegraphics[width=5cm,
height=3cm]{/Introduction/underground-station-a-6-s.png}
 \includegraphics[width=5cm,
height=3cm]{/Introduction/underground-station-a-7-s.png}
 \includegraphics[width=5cm,
height=3cm]{/Introduction/underground-station-a-8-s.png}
 \includegraphics[width=5cm,
height=3cm]{/Introduction/underground-station-a-9-s.png}
 \includegraphics[width=5cm,
height=3cm]{/Introduction/underground-station-a-10-s.png}
 %\includegraphics[width=8cm]{/Introduction/underground-station-a-11.png}
 \caption{A video sequence (from left to right, top to bottom) showing a man who
fell, was robbed, and hit by the train. Blue and red colors highlight the main
actor (drunk 	man) and the robber, while the train is indicated by green.}
 \label{fig:chp-intr:underground-sweden}
\end{figure}

\begin{figure}
 \centering
 \includegraphics[width=4cm]{/Introduction/Small/car-accident-a-1_s.png}
 \includegraphics[width=4cm]{/Introduction/Small/car-accident-a-2_s.png}
 \includegraphics[width=4cm]{/Introduction/Small/car-accident-a-3_s.png}
 \includegraphics[width=4cm]{/Introduction/Small/car-accident-a-4_s.png}
 \includegraphics[width=4cm]{/Introduction/Small/car-accident-b-1_s.png}
 \includegraphics[width=4cm]{/Introduction/Small/car-accident-b-2_s.png}
 \includegraphics[width=4cm]{/Introduction/Small/car-accident-b-3_s.png}
 \includegraphics[width=4cm]{/Introduction/Small/car-accident-b-4_s.png}
 \includegraphics[width=4cm]{/Introduction/Small/car-accident-d-1_s.png}
 \includegraphics[width=4cm]{/Introduction/Small/car-accident-d-2_s.png}
 \includegraphics[width=4cm]{/Introduction/Small/car-accident-d-3_s.png}
 \includegraphics[width=4cm]{/Introduction/Small/car-accident-d-4_s.png}
 \caption{Three car accidents in different circumstances recorded by CCTV
cameras. Sequences go from left to right.}
 \label{fig:chp-intr:car-accidents}
\end{figure}

\clearpage

\section{Problem Description}
In large CCTV installations, with some hundreds of cameras, only a small
fraction are ever watched. \cite{Dee:2007:HowClose} Surveillance systems have
been gained more emphasis in the world, especially since critical events, such
as terrorism attacks. The use of technology had improved, but improving hardware
is not the only alternative to face these issues. In United Kingdom, some
researchers showed that CCTV schemes in cities, town centers, and public housing
measured big ranges of crime types and only a small part of them where
studied, concluding that these schemes does not have a significant effect on
crime. \cite{Farrington:2007:EffectsCCTV,
Farrington:2008:EffectsCCTV} 

Eventually, one of the biggest problems is the fact to use human operators to
monitor surveillance cameras. Look at screens some hours, almost continuously,
is a difficult task which may have health impacts or even, create
controversy. Dee \etal \cite{Dee:2007:HowClose} collected reports to show that
the ratio between the operator and the number of CCTV cameras that has to watch
lies along $1:4$,$1:16$, $1:30$, and $1:78$. Inclusively, in some
installations, every camera needs to be watched at least once a day, but just a
few are monitored in real time. Whilst, Wallace \etal
\cite{Wallace:1998:CCTV} recommended that human operators may monitor no more
than $16$ cameras effectively in 30 minutes and, for health reasons, a break is
required every hour. Nowadays, display screen equipment is now a part for many
members of security staff, and those who regularly work for long periods on such
equipment may potentially suffer from a range of problems including, but not
limited to, eyestrain, musculo-skeletal problems, or stress.
\cite{HMPrison:2000:DisplayScreen} And considering social issues, operatives
frequently decide which cameras to monitor based on the appearance rather than
the behavior itself \cite{McCahill:2003:CCTV}. Therefore, exists a notion to
show that computational process to save community's security using devices,
already putting in work, is still in progress.

\section{Motivation}
In 1990s, the interest to use surveillance images in problems to help
computer assistance has gained importance. \cite{Rosin:1991:ImageInterpretation}
Good examples were projects like CROMATICA
\footnote{\url{
http://cordis.europa.eu/telematics/tap_transport/research/projects/cromatica.htm
l}}, PRISMATICA\footnote{\url{
http://www.transport-research.info/web/projects/project_details.cfm?ID=13699}},
and ADVISOR \footnote{\url{
http://www-sop.inria.fr/orion/ADVISOR/}} founded by the European Union to
monitor people in public environments.

Modeling the lack of the process to analyze surveillance videos as a
problem where the monitor needs to watch for uncommon occurrences, is the
same to identify abnormal events in surveillance systems. Although, for crime
prevention is not a direct application, it would help to prevent future events
to preserve security, or even more, to learn new patterns according to the
objects' behavior which are observed.

During all this time, many strategies to solve this problem have been
recognized, each one differing on features to take in consideration.
For instance, those which deals at a high-level and low-level processing.
\cite{Dee:2007:HowClose} High-level involves a understanding and integrates each
part within the scene, such as trajectory analysis, where a moving object is
considered to be an amorphous blob that is tracked. Even tough, tracking gives
an useful behavior (and/or contextual) model, computationally it is only
suitable for scenes with few objects (e.g. traffic monitoring
\cite{Kamijo:2000:TrafficMonitoring}), but impractical in crowded or complex
scenarios. Low-level processing extracts features (some of which are motion and
texture) to analyze the activity pattern of each pixel. Because this activity
cannot be used for a behavioral understanding, its applicability is reduced to
detection of local-temporal phenomena.

Although, as could be seen in \sct{chp:related-work}, low-level processing is
feasible to work in real time and adaptively, detection of anomal events is
a developing area which requires more attention to discover new approaches or
improve those already used. Humans are not good to detect anomalies, but
machines can help to improve this fact.

\section{Objectives and Contributions}
 \subsection{General Objective}
 This work aims to present an approach to automatically detect abnormal events
in surveillance videos exploring low-level features to exploit contextual
information adding a feedback process. The main goal is to look into new
frontiers in such a way that it improves the pixel-level \footnote{
Currently, an abnormal event is recognized at two levels. 1) Frame-level points
out what frames contain abnormal events, and 2) pixel-level shows pixels
that belong to an anomaly.} accuracy criterion.
\subsection{Specific Objectives}
 \begin{itemize}
  \item Explore spatio-temporal features, as well as, statistical frameworks to
describe the behavior model.
  \item Extract and learn, continuously, contextual information from incoming
stream of video sequences.
  \item Detect abnormal events in video surveillance sequences.
  \item Use statistical inference to determine whether or not an anomaly has
been occurred in a pixel-level manner.
	\item Update the model incrementally.
	\item Test, compare, and analyze results.
 \end{itemize}

 \subsection{Contributions}
 Unlike other works, a semi-supervised method will be explored into the update
process in order to improve the accuracy stated in the literature. If
successful, it will be a great contribution to the state-of-the-art. 

\section{Outline}
Chapter~\ref{chp:related-work} discuss the state-of-the-art as well as terms
and definitions, whilst Chapter~\ref{chp:methodology} 
describes the proposed approach in a general way. Chapter~\ref{chp:schedule}
presents the schedule with principal activities to be developed to get the
expected results, and finally, conclusions are shown in
Chapter~\ref{chp:conclusions}.
